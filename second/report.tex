\documentclass[a4paper,12pt]{ctexart} % 使用 ctexart 文档类,原生支持中文

% ==================== 宏包引入 ====================
\usepackage{geometry}       % 页面边距设置
\usepackage{graphicx}       % 插入图片
\usepackage{float}          % 图片位置强制固定
\usepackage{booktabs}       % 三线表
\usepackage{amsmath}        % 数学公式
\usepackage{listings}       % 代码块
\usepackage{xcolor}         % 颜色支持
\usepackage{hyperref}       % 目录跳转和超链接
\usepackage{caption}        % 图表标题格式
\usepackage{subcaption}

% ==================== 页面设置 ====================
\geometry{left=2.5cm, right=2.5cm, top=2.5cm, bottom=2.5cm}

% ==================== 代码高亮设置 ====================
\definecolor{codegreen}{rgb}{0,0.6,0}
\definecolor{codegray}{rgb}{0.5,0.5,0.5}
\definecolor{codepurple}{rgb}{0.58,0,0.82}
\definecolor{backcolour}{rgb}{0.95,0.95,0.92}

\lstdefinestyle{mystyle}{
    backgroundcolor=\color{backcolour},   
    commentstyle=\color{codegreen},
    keywordstyle=\color{magenta},
    numberstyle=\tiny\color{codegray},
    stringstyle=\color{codepurple},
    basicstyle=\ttfamily\footnotesize,
    breakatwhitespace=false,         
    breaklines=true,
    postbreak=\mbox{\textcolor{red}{$\hookrightarrow$}\space},
    captionpos=b,                    
    keepspaces=true,                 
    numbers=left,                    
    numbersep=5pt,                  
    showspaces=false,                
    showstringspaces=false,
    showtabs=false,                  
    tabsize=2,
    columns=flexible
}
\lstset{style=mystyle, language=Python}

% ==================== 文档信息 ====================
\title{\textbf{自然语言处理实验报告}\\ \large 基于RNN与CNN的词向量训练与对比分析}
\author{姓名:张宗桪 \\ 学号:2023K8009991013 \\ 专业:人工智能}
\date{\today}

% ==================== 正文开始 ====================
\begin{document}

\pagestyle{plain} % 页面顶部不显示标注,仅在底部显示页码

% 封面
\maketitle
\thispagestyle{empty} % 封面不显示页码
\newpage

% 目录
\tableofcontents
\newpage

% -------------------- 第一部分 --------------------
\section{实验目的与任务}

本实验旨在通过构建神经网络语言模型,利用《人民日报》语料库训练汉语词向量,并对比分析循环神经网络(RNN)与卷积神经网络(CNN)在特征提取和词向量生成上的差异。

主要任务包括:
\begin{enumerate}
    \item 数据预处理:清洗语料,构建词表,将文本转换为索引序列。
    \item 模型构建:分别搭建 RNN 和 CNN 模型。
    \item 词向量提取:从 Embedding 层获取词向量。
    \item 对比分析:计算并对比两个模型生成的词向量在语义相似度上的表现。
\end{enumerate}

% -------------------- 第二部分 --------------------
\section{实验原理与设置}

\subsection{模型架构}

\subsubsection{RNN 模型}
RNN 模型通过捕捉序列的时间依赖性来学习语义。本实验采用单向 RNN,结构如下:
\begin{itemize}
    \item \textbf{Embedding 层}:将词索引映射为稠密向量。
    \item \textbf{RNN 层}:处理输入序列,保留上下文信息。
    \item \textbf{全连接层}:将最后一个时间步的隐状态映射为词表概率分布。
\end{itemize}
\begin{figure}[H]
    \centering
    \includegraphics[width=0.6\textwidth]{./figures/RNN.png}
    \caption{RNN 模型结构示意图}
\end{figure}
\subsubsection{CNN 模型 }
CNN 模型通过滑动窗口提取局部特征。本实验结构如下:
\begin{itemize}
    \item \textbf{Embedding 层}:同上。
    \item \textbf{Conv1d 层}:一维卷积,核大小为 2 。
    \item \textbf{Max Pooling 层}:全局最大池化,提取最显著特征。
    \item \textbf{全连接层}:输出预测概率。
\end{itemize}
\begin{figure}[H]
    \centering
    \includegraphics[width=0.6\textwidth]{./figures/CNN.png}
    \caption{CNN 模型结构示意图}
\end{figure}
\subsection{实验参数设置}
\begin{table}[H]
    \centering
    \caption{实验超参数设置}
    \begin{tabular}{lc}
        \toprule
        \textbf{参数名} & \textbf{数值} \\
        \midrule
        词表大小  & 1000 \\
        词向量维度  & 15 \\
        输入序列长度  & 5 \\
        批大小  & 64 \\
        学习率  & 0.001 \\
        训练轮数  & 30 \\
        \bottomrule
    \end{tabular}
\end{table}

% -------------------- 第三部分 --------------------
\section{实验步骤与具体实现}

\subsection{数据预处理}
利用 Python 对《人民日报》1998年1月语料进行清洗。
\begin{itemize}
    \item \textbf{去噪}:剔除词性标记、数字、标点符号及非中文字符,仅保留纯汉字实词。
    \item \textbf{截断}:根据词频统计,保留前 1000 个高频词,其余标记为 \texttt{<UNK>}。
    \item \textbf{数据集构建}:采用滑动窗口法,取前 5 个词预测第 6 个词。
\end{itemize}

\subsection{核心代码实现}
下为 RNN 模型定义的关键代码片段:

\begin{lstlisting}
class RNNModel(nn.Module):
    def __init__(self, vocab_size, embed_dim, hidden_dim):
        super(RNNModel, self).__init__()
        self.embedding = nn.Embedding(vocab_size, embed_dim)
        self.rnn = nn.RNN(embed_dim, hidden_dim, batch_first=True)
        self.fc = nn.Linear(hidden_dim, vocab_size)

    def forward(self, x):
        embeds = self.embedding(x)
        out, _ = self.rnn(embeds)
        return self.fc(out[:, -1, :])
\end{lstlisting}
下为 CNN 模型定义的关键代码片段:
\begin{lstlisting}
class CNNModel(nn.Module):
    def __init__(self, vocab_size, embed_dim, seq_len):
        super(CNNModel, self).__init__()
        self.embedding = nn.Embedding(vocab_size, embed_dim)
        self.conv1 = nn.Conv1d(in_channels=embed_dim, out_channels=32, kernel_size=2)       
        self.pool = nn.AdaptiveMaxPool1d(1)        
        self.fc = nn.Linear(32, vocab_size)
    def forward(self, x):
        embeds = self.embedding(x)
        embeds = embeds.permute(0, 2, 1)
        x = torch.relu(self.conv1(embeds))
        x = self.pool(x).squeeze(-1)   
        return self.fc(x)
\end{lstlisting}
% -------------------- 第四部分 --------------------
\section{实验结果与分析}
\section{训练损失曲线}
\begin{figure}[H]
    \centering
    \begin{subfigure}{0.45\textwidth}
        \centering
        \includegraphics[width=\textwidth]{figures/CNN_loss_curve.png}
        \caption{CNN 训练损失曲线}
        \label{fig:sub1}
    \end{subfigure}
    \hfill
    \begin{subfigure}{0.45\textwidth}
        \centering
        \includegraphics[width=\textwidth]{figures/RNN_loss_curve.png}
        \caption{RNN 训练损失曲线}
        \label{fig:sub2}
    \end{subfigure}
    \caption{训练损失曲线对比}
    \label{fig:overall}
\end{figure}
在训练30批次的情况下,RNN的损失更低,故认定RNN的词向量效果更好。
\subsection{词向量展示}

\begin{table}[H]
    \centering
    \caption{CNN与RNN模型词向量示例(前8维)}
    \small
    \begin{tabular}{c|cccccccc}
        \toprule
        \textbf{词} & \textbf{$d_1$} & \textbf{$d_2$} & \textbf{$d_3$} & \textbf{$d_4$} & \textbf{$d_5$} & \textbf{$d_6$} & \textbf{$d_7$} & \textbf{$d_8$} \\
        \midrule
        \multicolumn{9}{c}{\textit{CNN 模型}} \\
        \midrule
        中国 & -1.02 & 0.74 & 0.56 & -0.32 & -0.55 & -1.07 & -0.86 & -0.67 \\
        \midrule
        \multicolumn{9}{c}{\textit{RNN 模型}} \\
        \midrule
        中国 & 0.17 & 1.34 & 0.77 & -2.23 & -0.17 & -2.03 & -0.28 & 0.93 \\
        \bottomrule
    \end{tabular}
\end{table}
\subsection{词向量对比分析}
我们选用例如“中国”、“经济”、“发展”等高频词进行对比分析。
\end{document}