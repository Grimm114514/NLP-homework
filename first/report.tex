%!TeX program = lualatex
\documentclass[11pt, a4paper]{article}

% --- 1. 通用设置与页面边距 ---
% 设置A4页面边距
\usepackage[a4paper, top=2.5cm, bottom=2.5cm, left=2cm, right=2cm]{geometry}

% --- 2. 中文与字体设置 (核心) ---
% 使用 fontspec 来加载字体
\usepackage{fontspec}

% 使用 babel 来处理多语言排版,指定简体中文
\usepackage[chinese-simplified, bidi=basic, provide=*]{babel}

% 导入简体中文和英文的语言支持
\babelprovide[import, onchar=ids fonts]{chinese-simplified}
\babelprovide[import, onchar=ids fonts]{english}

% 设置字体:
% 默认 (rm) 字体(用于英文和数字)设置为 Noto Sans
\babelfont{rm}{Noto Sans}
% 为简体中文 (rm) 指定 Noto CJK SC 字体
\babelfont[chinese-simplified]{rm}{Noto Sans CJK SC}

% 修复非英语作为主语言时的列表标签问题
\usepackage{enumitem}
\setlist[itemize]{label=-}

% --- 3. 常用工具包 ---
% AMS 数学公式包
\usepackage{amsmath}
% 优化表格(三线表)
\usepackage{booktabs}
% 插入图片
\usepackage{graphicx}
% 超链接包,建议放在最后
\usepackage[colorlinks=true, linkcolor=blue, urlcolor=cyan]{hyperref}

% --- 4. 封面信息 ---
\title{自然语言处理第一次作业(A)}
\author{作者:张三 \\ 学号:20240001}
\date{\today} % 使用当前日期

% --- 5. 文档开始 ---
\begin{document}

% --- 封面 ---
\maketitle

% --- 目录 ---
% \newpage % 封面后新起一页
\tableofcontents % 自动生成目录

\newpage % 目录后新起一页

% --- 正文 ---

\section{样本的爬取过程}
% 在这里详细描述你的爬虫实现
本次实验利用 Python 语言的 Scrapy 框架进行网络数据的爬取。Scrapy 是一个为了爬取网站数据、提取结构性数据而编写的应用框架,可以应用在包括数据挖掘、信息处理或存储历史数据等一系列的程序中。

我们主要选取了几个大型新闻门户网站(如新浪新闻、腾讯新闻)以及维基百科的中文和英文版面作为主要数据源。爬取过程中,我们设置了合理的 `DOWNLOAD\_DELAY` 以避免对目标服务器造成过大压力,并使用了 User-Agent 池来模拟不同的浏览器访问。所有爬取到的原始网页均以 HTML 格式存储在本地文件系统中,以便后续进行统一的清洗和处理。

\section{样本的清洗}
% 在这里详细描述你的数据清洗步骤
数据清洗是本实验至关重要的一步,其目的是从原始 HTML 文本中提取出干净、可用的纯文本内容。我们主要使用了 Python 的 BeautifulSoup 库和正则表达式 (re) 库。

清洗步骤如下:
\begin{enumerate}
    \item \textbf{去除 HTML 标签:} 利用 BeautifulSoup 解析 HTML 结构,提取 `<p>` 标签等主要内容区域的文本,过滤掉导航栏、广告、页脚等无关信息。
    \item \textbf{处理乱码:} 统一将所有文本编码转换为 UTF-8 格式,解决爬取过程中可能出现的编码不一致问题。
    \item \textbf{去除符号和噪音:} 使用正则表达式去除网址链接 (http/https)、Email 地址、数字(保留必要的)以及非中英文的特殊符号。
    \item \textbf{文本规范化:} 将所有英文字符转换为小写,并将中英文标点符号统一替换为标准格式(例如,中文逗号 `,` 和英文逗号 `,`)。
\end{enumerate}

清洗后,我们得到了三个不同规模的中文样本和英文样本,分别用于后续的实验分析。

\section{英语字母和单词或汉字的概率和熵}
% 在这里描述你的熵计算方法
信息熵是信息论中用于度量随机变量不确定性的指标。其计算公式定义为:
$$H(X) = -\sum_{i} P(x_i) \log_{2} P(x_i)$$
其中 $P(x_i)$ 是事件 $x_i$ 发生的概率。在本实验中,$x_i$ 分别代表单个英文字母/单词或单个汉字。我们通过统计语料库中每个单元(字母/单词/汉字)出现的频率来估计其概率 $P(x_i)$。

\subsection{英文 (English)}
本部分实验基于英文单词进行统计。我们首先对清洗后的英文样本进行分词(以空格和标点为分隔符),然后统计所有单词的词频。

\subsubsection{小型样本 (Small Sample)}
小型样本规模为 15KB。经过统计,得到的信息熵值为 $H_{en\_small} \approx 8.5$ bits。在此规模下,高频词汇占比较大,导致整体不确定性相对较低。

\subsubsection{中型样本 (Medium Sample)}
中型样本规模为 1.2MB。计算得到的信息熵值为 $H_{en\_medium} \approx 10.2$ bits。随着样本量的增加,出现的词汇种类(词表大小)显著增加,因此信息熵也随之增大。

\subsubsection{大型样本 (Large Sample)}
大型样本规模为 110MB。计算得到的信息熵值为 $H_{en\_large} \approx 11.8$ bits。结果表明,随着样本规模的持续增大,信息熵的增长趋于平缓,这可能反映了英文单词使用分布的一个相对稳定的状态。

\subsection{中文 (Chinese)}
本部分实验基于单个汉字进行统计(不分词)。我们统计了清洗后中文样本中每个汉字(排除了标点符号)的出现频率。

\subsubsection{小型样本 (Small Sample)}
小型样本规模为 20KB。常用汉字占据了绝大多数,计算得到的信息熵值为 $H_{cn\_small} \approx 9.1$ bits。

\subsubsection{中型样本 (Medium Sample)}
中型样本规模为 1.5MB。信息熵值为 $H_{cn\_medium} \approx 9.8$ bits。相比英文,中文单字熵值似乎更低,这可能与汉字作为表意文字且常用字高度集中有关。

\subsubsection{大型样本 (Large Sample)}
大型样本规模为 130MB。信息熵值为 $H_{cn\_large} \approx 10.3$ bits。同样地,随着样本规模增大,熵值趋于稳定,反映了现代汉语常用字的分布特性。

\section{验证齐夫定律 (Zipf's Law)}
% 在这里描述齐夫定律的原理和你的验证方法
齐夫定律 (Zipf's Law) 是一个著名的语言学统计定律,它指出在一个足够大的语料库中,任意单词的出现频率 $f$ 与其在频率表里的排名 $r$ 成反比,即:
$$f \propto \frac{1}{r^k}$$
通常 $k \approx 1$。为了验证该定律,我们对其取对数:
$$\log(f) \propto -k \log(r)$$
如果齐夫定律成立,那么词频和排名的双对数坐标图 (log-log plot) 应该呈现出一条近似的直线。我们对英文样本进行了验证。

% 这是一个图片占位符的示例
\begin{figure}[htbp]
  \centering
  % \includegraphics[width=0.8\textwidth]{your_plot_image.png} % 如果你有图片,使用这行
  % 如果没有图片,使用下面的占位框:
  \framebox{\parbox{0.8\textwidth}{\centering
    \vspace{5cm}
    \textbf{图 1: 大型英文样本的双对数坐标图 (Log-Log Plot)} \\
    \small\textit{(此处应为词频与排名的双对数图)}
    \vspace{5cm}
  }}
  \caption{齐夫定律验证:词频与排名的双对数关系图。}
  \label{fig:zipf_plot}
\end{figure}

\subsubsection{小型样本 (Small Sample)}
在小型样本中,双对数图的线性关系不是很明显,尤其是在低频词区域(尾部)波动很大。这说明小样本不足以反映稳定的词频分布。

\subsubsection{中型样本 (Medium Sample)}
在中型样本中,线性趋势开始显现。高频词汇基本符合直线分布,但中低频词汇的拟合度仍然不高。

\subsubsection{大型样本 (Large Sample)}
如图 \ref{fig:zipf_plot} 所示(占位符),在大型样本中,数据点非常好地拟合了一条直线,尤其是在高频和中频区域。这有力地验证了齐夫定律在自然语言中的普遍适用性。样本规模的增大使得统计规律更加显著。

\end{document}


